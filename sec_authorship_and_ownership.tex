%!TEX root = Pirate_Copyright_Code.tex

%\section{Authorship and Ownership}
\section{Urheberschaft und Inhaberschaft}
\begin{contract}

%=================================================%

%#English Version#
%\Paragraph{title=Authorship}\label{Par:Authorship}
%\Sentence The author of a work is the natural person or group of natural persons who
%created it.

\Paragraph{title=Urheberschaft}\label{Par:Authorship}
\Sentence Der Urheber eines Werke ist die nat�rliche Person oder Gruppe von nat�rlichen Personen, welche es erschufen.

%=================================================%

%#Englisch Version#
%\Paragraph{title=Moral rights}\label{Par:MoralRights}
%\Sentence The author of the work has the moral rights.

\Paragraph{title=Ideelle Rechte}\label{Par:MoralRights}
\Sentence Der Urheber des Werkes hat ideelle Rechte.

%#English Version#
%\Sentence Moral rights cannot be assigned.

\Sentence Ideelle Rechte sind nicht �bertragbar.


%=================================================%

%#English Version#
%\Paragraph{title=Economic rights}\label{Par:EconomicRights}
%\Sentence The owner of the economic rights in a work is its author.

\Paragraph{title=Verwertungsrechte}\label{Par:EconomicRights}
\Sentence Der Inhaber der Verwertungsrechte an einem Werk ist dessen Urheber.

%#English Version#
%\Sentence  Subject to the restrictions of article~\refParagraphN{Par:Limits}, the economic rights in a work may
%be assigned, licensed and passed by inheritance, in whole or in part.

\Sentence Unter den Bedingungen des Artikels~\refParagraphN{Par:Limits} k�nnen die Verwertungsrechte an einem Werk ganz oder teilweise �bertragen, lizensiert und vererbt werden.

%#English Version#
%\Sentence If the author has assigned economic rights, he shall nonetheless have a right to an adequate part of the remuneration on the basis of the
%provisions in articles~\refParagraphN{Par:UsesForThePurposeOfFreedomOfExpressionAndInformation}, \refParagraphN{Par:UsesPermittedToPromoteSocialPoliticalAndCulturalObjectives}, \refParagraphN{Par:UsesForThePurposeOfEnhancingCompetition} and \refParagraphN{Par:FurtherLimitations}.

\Sentence Sollte der Urheber seine Verwertungsrechte �bertragen haben, steht ihm dennoch das Recht auf einen angemessenen Teil der Entsch�digung aus den Abgaben auf Basis der Artikel~\refParagraphN{Par:UsesForThePurposeOfFreedomOfExpressionAndInformation}, \refParagraphN{Par:UsesPermittedToPromoteSocialPoliticalAndCulturalObjectives}, \refParagraphN{Par:UsesForThePurposeOfEnhancingCompetition} and \refParagraphN{Par:FurtherLimitations} zu.

%E
%\Sentence An assignment is not valid unless it is made in writing.

\Sentence Eine �bertragung ist nicht g�ltig solange sie nicht schriftlich erfolgt.

%=================================================%

%E
%\Paragraph{title=Limits}\label{Par:Limits}
%\Sentence If the contract by which the author assigns or exclusively licenses the
%economic rights in his work does not adequately specify 
%\begin{enumerate}[nolistsep,label=\alph*.,labelindent=1em,labelsep=*,align=left,beginpenalty=10000,endpenalty=10000]
%	\item the amount of the author's remuneration, 
%	\item the geographical scope, 
%	\item the mode of exploitation and 
%	\item the duration of the grant, 
%\end{enumerate}
%the extent of the grant shall be determined
%in accordance with the purpose envisaged in making the grant.

\Paragraph{title=Einschr�nkungen}\label{Par:Limits}
\Sentence Wenn der Vertrag, in welchem der Urheber seine Verwertungsrechte �bertr�gt oder exklusiv lizensiert  
\begin{enumerate}[nolistsep,label=\alph*.,labelindent=1em,labelsep=*,align=left,beginpenalty=10000,endpenalty=10000]
	\item die H�he des Entgelts des Urhebers,
	\item den geographischen Umfang,
	\item die Art und Weise der Verwertung und 
	\item die Dauer der �berlassung
\end{enumerate}
nicht angemessen spezifiziert, wird das Ausma� der �berlassung in �bereinstimmung mit dem Zweck, der durch die �berlassung beabsichtigt war, festgelegt.

%=================================================%

%E
%\Paragraph{title=Works made in the course of employment}\label{Par:WorksMadeInTheCourseOfEmployment}
%\Sentence Unless otherwise agreed, the economic rights in a work created by the author
%in the execution of his duties or following instructions given by his employer
%are deemed to be assigned to the employer.

\Paragraph{title=Werke im Rahmen eines Arbeitsverh�ltnisses}\label{Par:WorksMadeInTheCourseOfEmployment}
\Sentence Soweit nicht anders vereinbart, fallen die Verwertunsgrechte an einem Werk, welches vom Urheber im Rahmen der Aufgaben seines Arbeitsverh�ltnisses oder der Anweisungen seines Arbeitgebers entstanden, an den Arbeitgeber. 

%=================================================%

%E
%\Paragraph{title=Works made on commission}\label{Par:WorksMadeOncommission}
%\Sentence Unless otherwise agreed, the use of a work by the commissioner of that work
%is authorised to the extent necessary to achieve the purposes for which the
%commission was evidently made.

\Paragraph{title=Werke auf Bestellung}\label{Par:WorksMadeOncommission}
\Sentence Soweit nicht anders vereinbart, ist der Besteller eines Werkes zur Nutzung berechtigt die notwendig ist, um den nachweislichen Zweck der Bestellung zu erf�llen. 

\end{contract}
