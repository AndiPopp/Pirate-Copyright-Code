%!TEX root = Pirate_Copyright_Code.tex

%Language: de_DE

\section{Der urspr�ngliche WITTEM Code}

Diese Arbeit ist ein Derivat des "`European copyright code"', welcher im April~2010 von der WITTEM-Gruppe ver�ffentlicht wurde. Das WITTEM-Projekt wurde 2002 von Urheberrechts-Wissenschaftlern aus mehreren europ�ischenL�ndern gegr�ndet. Das Ziel war die Erstellung eines Urheberrechts-Entwurfs f�r eine vereinte Europ�ische Union.

Die Mitglieder der WITTEM-Gruppe beschreiben ihr Ziel wie folgt:

\begin{quote}
Das Ziel des Wittem-Projekts und des Gesetzentwurfs ist die F�rderung von Transparenz und Konsistenz im europ�ischen Urheberrecht. Die Mitglieder der Wittem-Gruppe teilen die Besorgnis, dass es dem Prozess der Urheberrechtsgesetzgebung auf europ�ischer Ebene an Transparenz mangelt und das akademische Stimmen all zu oft �berh�rt werden. Die Gruppe glaubt, dass ein von Rechts-Wissenschaftlern erstellter europ�ischer Urheberrechtsentwurf als Modell oder Referenz f�r zuk�nftige Harmonisierung oder Vereinheitlichung des Urheberrechts auf europ�ischer Ebene dienen kann. Dennoch vertritt die Gruppe keine Position, �ber die Notwendigkeit der Einf�hrung eines einheitlichen europ�ischen Gesetzesrahmens.\footnote{cf.~\textit{European copyright code}~p.~5 -- http://www.copyrightcode.eu -- Inoffizielle �bersetzung}
\end{quote}

Wir -- die Autoren -- als Mitglieder der Piratenpartei Deutschland waren sehr beeindruckt von diesen Zielen. Die Integration von akademischem Know-How und Expertenwissen war immer Teil des Politikstils der Piratenpartei. Transparenz ist eines der Hauptziele ihres Parteiprogramms. W�hrend der Lekt�re des konkreten Inhalts des Gesetzentwurfs, wurden wir von dessen Einfachheit und guter Struktur �berzeugt.

Die Piratenparteien verstehen sich selbst als internationale -- und damit nat�rlich europ�ische -- Bewegung und die Reform des Urheberrechts als eines ihrer gemeinsamen Ziele. Dennoch ist ihr Interesse eher politischer als juristischer Natur. 
Wir glauben, dass ein Urheberrechtsenwurf nach Piratenart, welcher in allen europ�ischen Staaten anwendbar ist, 
Wir glauben dass ein Entwurf f�r ein Urheberrecht im Sinne der Piraten, welches in allen europ�ischen Staaten eingef�hrt werden kann, diesem Ziel dienlich ist. Da die WITTEM-Gruppe ihren Entwurf als Modell zur Weiterentwicklung betrachtet und wir diesen als modernes und akademisch wohl durchdachtes Werk betrachten, entschieden wir uns daf�r unseren eigenen Entwurf darauf aufzubauen.
