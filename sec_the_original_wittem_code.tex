%!TEX root = Pirate_Copyright_Code.tex
\section{The Original WITTEM Code}

This work is a derivate of the ``European copyright code'' published by the WITTEM Group in April~2010. The WITTEM~project was founded in 2002 by copyright scholars across the European Union. The goal was to create a draft for a unified European copyright code. 

The members of the WITTEM group describe their goals as the following:

\begin{quote}
	The aim of the Wittem Project and this Code is to promote transparency and
consistency in European copyright law. The members of the Wittem Group
share a concern that the process of copyright law making at the European
level lacks transparency and that the voice of academia all too often remains
unheard. The Group believes that a European Copyright Code drafted by legal
scholars might serve as a model or reference tool for future harmonization
or unification of copyright at the European level. Nevertheless, the Group
does not take a position on the desirability as such of introducing a unified
European legal framework.\footnote{cf.~\textit{European copyright code}~p.~5 -- http://www.copyrightcode.eu}
\end{quote}

We -- the authors -- as members of the Pirate Party of Germany were very impressed by these aims. The integration of academic and expert knowledge was always part of the Pirate Party's style of politics. Transparency is one of the main goals of its political platform. While reading the code's actual content we were convinced by its simplicity and well organized structure. 

The Pirate Parties think of themselves as an international and of course a European movement with the reformation of copyright law as one of their common goals. However, they have a political interest in the matter rather than a legal or academic one. We think that for this goal a draft for a pirate like copyright, which is implementable in all European countries, is useful. Since the WITTEM group sees its draft as model for further development and we appreciate it as a modern and academicly highly sophisticated work we decided to build our own draft based on this. 