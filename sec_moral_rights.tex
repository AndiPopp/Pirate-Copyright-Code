%!TEX root = Pirate_Copyright_Code.tex
\section{Moral Rights}
\begin{contract}
\Paragraph{title=General}\label{Par:MoralRightsGeneral}
\Sentence The moral rights in a work are the rights of divulgation, attribution and
integrity, as provided for in articles~\refParagraphN{Par:RightOfDivulgation}, \refParagraphN{Par:RightOfAttribution} and \refParagraphN{Par:RightOfIntegrity}.

%=================================================%
\Paragraph{title=Right of divulgation}\label{Par:RightOfDivulgation}
\Sentence The right of divulgation is the right to decide whether, and how the work is disclosed for the first time.

\Sentence This right shall last for the life of the author.

%=================================================%
\Paragraph{title=Right of attribution}\label{Par:RightOfAttribution}
\Sentence  The right of attribution comprises:
\begin{legalenum}
	\item the right to be identified as the author, including the right to choose
				the manner of identification, and the right, if the author so decides,
				to remain unidentified.
	\item the right to require that the name or title which the author has given to
				the work being indicated.
\end{legalenum}

\Sentence This right shall last for the life of the author and until [...] years after his
death.\Sentence The legal successor as defined by the laws on inheritance is
entitled to exercise the rights after the death of the author.

%=================================================%
\Paragraph{title=Right of integrity}\label{Par:RightOfIntegrity}
\Sentence  The right of integrity is the right to object to any distortion, mutilation
or other modification, or other derogatory action in relation to the work,
which would be prejudicial to the honour or reputation of the author.

\Sentence This right shall last for the life of the author and until [...] years after his
death.\Sentence The legal successor as defined by the laws on inheritance shall be
entitled to exercise the right after the death of the author.

%=================================================%
\Paragraph{title=Consent}\label{Par:Consent}
\Sentence The author can consent not to exercise his moral rights. \Sentence Such consent must be limited in scope, unequivocal and informed.

%=================================================%
\Paragraph{title=Interests of third parties}\label{Par:InterestsOfThirdParties}
\Sentence  The moral rights recognised in article~\refParagraphN{Par:MoralRightsGeneral} will not be enforced in situations 
where to do so would harm the legitimate interests of third parties to
an extent which is manifestly disproportionate to the interests of the
author.

\Sentence  After the author's death, the moral rights of attribution and integrity
shall only be exercised in a manner that takes into account the interests
in protecting the person of the deceased author, as well as the legitimate
interests of third parties.

\end{contract}