%!TEX root = Pirate_Copyright_Code.tex

%E
%\section{Moral Rights}

\section{Ideelle Rechte}
\begin{contract}

%=================================================%

%E
%\Paragraph{title=General}\label{Par:MoralRightsGeneral}
%\Sentence The moral rights in a work are the rights of divulgation, attribution and
%integrity, as provided for in articles~\refParagraphN{Par:RightOfDivulgation}, \refParagraphN{Par:RightOfAttribution} and \refParagraphN{Par:RightOfIntegrity}.

\Paragraph{title=Allgemeines}\label{Par:MoralRightsGeneral}
\Sentence Die moralischen Rechte an einem Werk sind die Rechte auf Erstver�ffentlichung, Namensnennung und Integrit�t, wie sie in den Artikel~\refParagraphN{Par:RightOfDivulgation}, \refParagraphN{Par:RightOfAttribution} und \refParagraphN{Par:RightOfIntegrity} festgehalten sind.


%=================================================%

%E
%\Paragraph{title=Right of divulgation}\label{Par:RightOfDivulgation}
%\Sentence The right of divulgation is the right to decide whether, and how the work is disclosed for the first time.

\Paragraph{title=Recht auf Erstver�ffentlichung}\label{Par:RightOfDivulgation}
\Sentence Das Recht auf Erstver�ffentlichung ist das Recht zu entscheiden ob und wie das Werk zum ersten Mal ver�ffentlicht wird.

%E
%\Sentence This right shall last for the life of the author.

\Sentence Dieses Recht besteht bis zum Tod des Urhebers.



%=================================================%

%E
%\Paragraph{title=Right of attribution}\label{Par:RightOfAttribution}
%\Sentence  The right of attribution comprises:
%\begin{legalenum}
%	\item the right to be identified as the author, including the right to choose
%				the manner of identification, and the right, if the author so decides,
%				to remain unidentified.
%	\item the right to require that the name or title which the author has given to
%				the work being indicated.
%\end{legalenum}

\Paragraph{title=Recht auf Namensnennung}\label{Par:RightOfAttribution}
\Sentence Das Recht auf Namensnennung besteht aus
\begin{legalenum}
	\item dem Recht als Urheber genannt zu werden, inclusive das Recht die Art und Weise dieser 
		Nennung zu w�hlen und dem Recht nicht genannt zu werden, sofern der Urheber dies w�nscht.
	\item dem Recht zu verlangen, dass auf den Name oder Titel, welchen der Urheber dem Werk 
		gegeben hat hingewiesen wird
\end{legalenum}

%E
%\Sentence This right shall last for the life of the author and until [...] years after his
%death.\Sentence The legal successor as defined by the laws on inheritance is
%entitled to exercise the rights after the death of the author.

\Sentence Diese Recht besteht bis zum Tod des Urhebers und [...] Jahre danach. \Sentence Der juristische Nachfolger ist durch die Erbschaftsgesetze bestimmt und ist berechtigt diese Rechte nach dem Tod des Autors auszu�ben.

%=================================================%

%E
%\Paragraph{title=Right of integrity}\label{Par:RightOfIntegrity}
%\Sentence  The right of integrity is the right to object to any distortion, mutilation
%or other modification, or other derogatory action in relation to the work,
%which would be prejudicial to the honour or reputation of the author.

\Paragraph{title=Recht auf Integrit�t}\label{Par:RightOfIntegrity}
\Sentence Das Recht auf Schutz vor Entstellung ist das Recht jeder Verf�lschung, Entstellung oder sonstigen Ver�nderung oder abwertenden Handlung in Bezug auf das Werk, welche der Ehre und dem Ruf des Urhebers abtr�glich sind, zu widersprechen.

%E
%\Sentence This right shall last for the life of the author and until [...] years after his
%death.\Sentence The legal successor as defined by the laws on inheritance is
%entitled to exercise the rights after the death of the author.

\Sentence Diese Recht besteht bis zum Tod des Urhebers und [...] Jahre danach. \Sentence Der juristische Nachfolger ist durch die Erbschaftsgesetze bestimmt und ist berechtigt diese Rechte nach dem Tod des Autors auszu�ben.

%=================================================%

%E
%\Paragraph{title=Consent}\label{Par:Consent}
%\Sentence The author can consent not to exercise his moral rights. \Sentence Such consent must be limited in scope, unequivocal and informed.

\Paragraph{title=Zustimmung}\label{Par:Consent}
\Sentence Der Urheber kann zustimmen seine ideellen Rechte nicht auszu�ben. \Sentence Eine solche Zustimmung muss von beschr�nktem Ausma�, unmissverst�ndlich und informiert sein.

%=================================================%

%E
%\Paragraph{title=Interests of third parties}\label{Par:InterestsOfThirdParties}
%\Sentence  The moral rights recognised in article~\refParagraphN{Par:MoralRightsGeneral} will not 
%be enforced in situations where to do so would harm the legitimate interests of third parties to
%an extent which is manifestly disproportionate to the interests of the
%author.

\Paragraph{title=Interesse von Dritten}\label{Par:InterestsOfThirdParties}
\Sentence  Die ideellen Rechte, welche im Artikeln~\refParagraphN{Par:MoralRightsGeneral} anerkannt werden, werden in Situation nicht durchgesetzt, in welchen dies die legitimen Interessen von Dritten in einem Ausma� beeintr�chtigen w�rde, welches offenkundig nicht im Verh�ltnis zum Interesse des Urhebers steht.

%E
%\Sentence  After the author's death, the moral rights of attribution and integrity
%shall only be exercised in a manner that takes into account the interests
%in protecting the person of the deceased author, as well as the legitimate
%interests of third parties.

\Sentence Nach dem Tod des Autors d�rfen die ideellen Rechte auf Namensnennung und Integrit�t nur in einem Umfang angewendet werden, welcher das Interesse die Person des verstorbenen Urhebers, sowie legitime Rechte Dritter ber�cksichtigt.

\end{contract}
