%!TEX root = Pirate_Copyright_Code.tex
\section{Economic Rights}
\begin{contract}

%=================================================%
\Paragraph{title=General}\label{Par:EconomicRightsGeneral}
\Sentence The economic rights in a work are the exclusive rights to authorize or 
prohibit the reproduction, distribution, rental, communication to the 
public and adaptation of the work, in whole or in part, as provided for in 
articles~\refParagraphN{Par:RightOfReproduction}, \refParagraphN{Par:RightOfDistribution}, \refParagraphN{Par:RightOfRental}, \refParagraphN{Par:RightOfCommunicationToThePublic} and \refParagraphN{Par:RightOfAdaption}.

\Sentence Economic rights apply to commercial use. Commercial use is every use that has the intention of generating monetary profit or aiding in the generation of monetary profit.

%=================================================%
\Paragraph{title=Claiming and duration of economic rights}\label{Par:ClaimingAndDurationOfEconomicRights}
\Sentence In order to exercise economic rights, the author has to attach a copyright claim to the published work. \Sentence A corresponding copyright declaration has to be published within appropriate time. 

\Sentence The copyright declaration has to contain:
\begin{legalenum}
	\item title
	\item type of work
	\item author
	\item date of publishing
	\item whether or not the right to  freely make derivative works is granted
	\item whether or not the right of free commercial use is granted
	\item whether or not the author obliges to publish all his source material
	\item special default conditions of use
\end{legalenum}

\Sentence Special default restrictions of use may not contradict other granted rights in the copyright declaration.

\Sentence Commercial use of a work, which violates the author's correctly claimed and declared economic rights is to be ceased on the author's request. 

\Sentence Economic rights expire 5~years after the date of publishing. The duration extends
\begin{legalenum}
	\item by 5~years if the right to  freely make derivative works is granted
	\item by 5~years if  the right of free commercial use is granted
	\item by 10~years if  the author obliges to publish all his source material
\end{legalenum}

%=================================================%
\Paragraph{title=Works excluded from claiming of economic rights}\label{Par:WorksExcludedFromClaimingOfEconomicRights}
\Sentence Economic rights cannot be claimed for the following works:
\begin{legalenum}
	\item works that are created during the working hours of a public employee
	\item works that are created by a public employee, whose job description contains the creation of the same types of work as the work at hand
	\item works that are directly publicly funded by more than 50\%
\end{legalenum}

%=================================================%
\Paragraph{title=Right of reproduction}\label{Par:RightOfReproduction}
\Sentence The right of reproduction is the right to reproduce the work in any manner 
or form, including temporary reproduction insofar as it has independent 
economic significance.

%=================================================%
\Paragraph{title=Right of distribution}\label{Par:RightOfDistribution}
\Sentence The right of distribution is the right to distribute to the public the original 
of the work or copies thereof.

\Sentence The right of distribution does not apply to the distribution of the original 
or any copy that has been put on the market by the holder of the copyright 
or with his consent.

%=================================================%
\Paragraph{title=Right of rental}\label{Par:RightOfRental}
\Sentence The right of rental is the right to make available the original of the work 
or copies thereof for use for a limited period of time for profit making 
purposes.

\Sentence The right of rental does not extend to the rental of buildings and works of 
applied art.

%=================================================%
\Paragraph{title=Right of communication to the public}\label{Par:RightOfCommunicationToThePublic}
\Sentence  The right of communication to the public is the right to communicate the 
work to the public, including but not limited to public performance, 
broadcasting, and making available to the public of the work in such a 
way that members of the public may access it from a place and at a time 
individually chosen by them.

\Sentence  A communication of a work shall be deemed to be to the public if it is 
intended for a plurality of persons, unless such persons are connected by 
personal relationship.

%=================================================%
\Paragraph{title=Right of adaptation}\label{Par:RightOfAdaption}
\Sentence The right of adaptation is the right to adapt, translate, arrange or otherwise 
alter the work.

%=================================================%
\Paragraph{title=Subsequent exploitation}\label{Par:SubsequentExploitation}
\Sentence Beginning from the expiration of formerly claimed and granted commercial rights until the author's death, the author is entitled to a minimum share of the profits generated by commercial exploitation of the work according to the formerly mentioned economic rights.

\Sentence The height of the share is decided by the proper public authorities.

\end{contract}