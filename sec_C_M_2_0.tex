\subsection{Political Goals for Version 2}

\subsubsection{Subsequent Exploitation}

Incremental economic rights duration is accompanied by a reform of the length of economic rights duration itself and the starting point of its expiration. Unlike the TRIPS treaty this reform uncouples economic rights duration from the authors life time. This however leads to the situation, that a commercial exploitation during the life time of the author is possible without the author benefiting from it. This might lead to a deincentivation of the author which is contrary to the copyright's goal.

The most important point in short duration of economic rights is to give the work back into public domain (at least partially). This mainly means the author (resp. other holder of the economic rights) should not be allowed to exclude anyone from using his work anymore. Note that since private use is already granted, we talk about commercial use here. Between the expiration of the economic rights and the death of the author, we put the work into the state of subsequent exploitation. In this time period nobody needs to ask the author's permission to use the work in a commercial context anymore but is obliged to give a share of his revenue to the author. This share should be relatively small, so that the economic possibilities to use the work are considerably better than during the economic rights duration. A public authority should be entitled to define this minimum share. Collection of these shares could for example be handled by a collecting society.

\subsubsection{Publicly Funded Works}



\subsection{Code Modifications for Version 2}

\subsubsection{Subsequent Exploitation}

The idea is implemented in the new Art.~\refParagraphN{Par:SubsequentExploitation}.

\subsubsection{Publicly Funded Works}