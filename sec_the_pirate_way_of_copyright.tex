%!TEX root = Pirate_Copyright_Code.tex
\section{The Pirate Way of Copyright}

%Englisch Version = While the work of the WITTEM group is scientificly good, the code itself is not suitable as a draft for a pirate copyright. The reason is the actual political premises the original code is build apon. The WITTEM group orients its work on the Bern convention\footnote{cf.~\url{http://en.wikipedia.org/wiki/Berne_Convention_for_the_Protection_of_Literary_and_Artistic_Works}} and the TRIPS agreement\footnote{cf.~\url{http://en.wikipedia.org/wiki/Agreement_on_Trade-Related_Aspects_of_Intellectual_Property_Rights}}. For the Pirate Parties however these international agreements are highly anachronistic and should not be preserved in their current form. Therefore, our draft for a copyright code focuses on the way copyright law should be in the information age not the way it has to be to satisfy the last milleniums international agreements.

Während die Arbeit der WITTEM-Gruppe aus wissenschaftlicher Sicht gut ist, eignet sich der Gesetzentwurf nicht als ein Piraten-Urheberrecht. Der Grund ist, dass der Gesetzentwurf auf den aktuellen politischen Voraussetzungen basiert. Die WITTEM-Gruppe orientiert sich bei ihrer Arbeit an der Berner Übereinkunft\footnote{cf.~\url{http://de.wikipedia.org/wiki/Berner_Übereinkunft_zum_Schutz_von_Werken_der_Literatur_und_Kunst}} und dem TRIPS-Abkommen\footnote{cf.~\url{http://de.wikipedia.org/wiki/TRIPS}}. Für die Piratenparteien sind diese internationalen Abkommen stark anachronistisch und sollten ihn ihrer jetzigen Form keinen Bestand mehr haben. Aus diesem Grund konzentriert sich unser Urheberrechts-Entwurf darauf, wie das Urheberrecht sein soll um für das Informationszeitalter fit zu sein, nicht wie es sein muss um veralteten Übereinkommen aus dem letzten Jahrtausend gerecht zu werden.

%Englisch version = We also want to experiment on the way to develop our own code with actually employing some aspects of the remix and open source culture. We therefore see this work as a remix of the original WITTEM code. We do not think about the code as completed paper but rather a developing idea like an actual open source project. We want to encourage interested people to draft their own remixes and contribute their ideas and remarks back to the Pirate Copyright Code.

Wir wollen auch mit der Art experimentieren, wie wir den Gesetzentwurf entwickeln, indem wir eine einige Aspekte der Remix- und Open-Source-Kultur nutzen. Aus diesem Grund sehen wir auch dieses Werk als Remix des ursprüngliche WITTEM-Entwurfs. Wir verstehen den Entwurf weniger als fertiges Papier, sondern viel mehr als eine sich entwickelnde Idee, ähnlich eines klassischen Open-Source-Projekts. Wir wollen Interessierte dazu ermutigen ihre eigenen Remixe zu entwerfen und mit ihren Ideen und Anmerkungen zum Pirate Copyright Code beizutragen.  

%English version = We therefore give version numbers with each code. First tier version numbers always imply the implementation of so far neglected political goals. Second tier version numbers imply various corrections of the code itself on the basis of the currently regarded political goals.

Aus diesem Grund versehen wir den Entwurf mit Versionsnummern. Versionsnummern der ersten Ebene implizieren die Implementierung bis dato nicht beachteter politischer Zieler. Versionsnummern zweiter Ordnung implizieren unterschiedliche Korrekturen des Entwurfs auf Basis der aktuell betrachteten Ziele. 
