\section{The Pirate Way of Copyright}

While the work of the WITTEM group is scientificly good, the code itself is not suitable as a draft for a pirate copyright. The reason is the actual political premises the original code is build apon. The WITTEM group orients its work on the Bern convention\footnote{cf.~\url{http://en.wikipedia.org/wiki/Berne_Convention_for_the_Protection_of_Literary_and_Artistic_Works}} and the TRIPS agreement\footnote{cf.~\url{http://en.wikipedia.org/wiki/Agreement_on_Trade-Related_Aspects_of_Intellectual_Property_Rights}}. For the Pirate Parties however these international agreements are highly anachronistic and should not be preserved in their current form. Therefore, our draft for a copyright code focuses on the way copyright law should be in the information age not the way it has to be to satisfy the last milleniums international agreements.

We also want to experiment on the way to develop our own code with actually employing some aspects of the remix and open source culture. We therefore see this work as a remix of the original WITTEM code. We do not think about the code as completed paper but rather a developing idea like an actual open source project. We want to encourage interested people to draft their own remixes and contribute their ideas and remarks back to the Pirate Copyright Code.

We therefore give version numbers with each code. First tier version numbers always imply the implementation of so far neglected political goals. Second tier version numbers imply various corrections of the code itself on the basis of the currently regarded political goals.